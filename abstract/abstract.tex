\documentclass[11pt,dvipdfmx,svgnames,a4paper,uplatex]{ujarticle}
% +++++++++++++++++++++++++++++++++++++++++++
% パッケージの導入
% +++++++++++++++++++++++++++++++++++++++++++
%
% ===========================================
% 原稿設定
% ===========================================
% 原稿のサイズ
\usepackage[top=25truemm,bottom=25truemm,left=25truemm,right=25truemm]{geometry}
\setlength\intextsep{0pt}
\setlength\textfloatsep{0pt}
\usepackage{relsize}
\usepackage{setspace}
\onehalfspacing
\usepackage[compact]{titlesec}
\titleformat{\section}{\normalfont\bfseries}{\thesection}{1em}{}  % \sectionのフォントサイズを本文と統一
\titlespacing*{\section}{0pt}{*0}{0pt}  % section前後のスペーシングをなくす
\pagestyle{empty}
% 一行あたり文字数と一ページあたり行数の指定
\makeatletter
\def\mojiparline#1{
  \newcounter{mpl}
  \setcounter{mpl}{#1}
  \@tempdima=\linewidth
  \advance\@tempdima by-\value{mpl}zw
  \addtocounter{mpl}{-1}
  \divide\@tempdima by \value{mpl}
  \advance\kanjiskip by\@tempdima
  \advance\parindent by\@tempdima
}
\makeatother
\def\linesparpage#1{
  \baselineskip=\textheight
  \divide\baselineskip by #1
}
%
% ===========================================
% 図・表関係
% ===========================================
\usepackage{graphicx}
\usepackage{wrapfig}  % 文章に囲まれた図の挿入
\graphicspath{  % \includegraphicsで参照するディレクトリ
  {../thesis/pics/}
}
%
% ===========================================
% 参考文献
% ===========================================
\usepackage[utf8]{inputenc} % Load 'inputenc' before biblatex.のWarningの回避.utf-8を使うためのパッケージ
\usepackage[backend=biber,style=phys,biblabel=brackets,pageranges=false,maxbibnames=3,doi=false]{biblatex}
\setlength\bibitemsep{0.0\baselineskip}
\addbibresource{../bib_textbooks.bib}  % 教科書など
\addbibresource{../bib_articles.bib}  % 論文など
%
% ===========================================
% 独自スタイルの導入
% ===========================================
\usepackage{../mystyle}
\hypersetup{pdfborder={0 0 0}}  % hyperlinkの青枠線を表示しない
\usepackage{setspace,caption} % Unknown document class (or package)のWarningの回避のためにmystyleのあとに書く
\captionsetup[figure]{font={stretch=0.8}}  % 図のキャプションのスペーシングを縮める

\renewcommand{\figurename}{図} % 2022年度からは研究室内で日本語表記に統一する
\renewcommand{\tablename}{表}
\usepackage{bxjalipsum}  % 日本語のダミー文章
\DeclareSourcemap{
  % 論文のタイトルを非表示(mystyle.styで論文のタイトル表示からダブルクォーテーションを削除,という指示をしているのでここに書く)
  \maps[datatype=bibtex]{
    \map{
      \pertype{article}
      \step[fieldset=title, null]
    }
  }
}

\usepackage{csquotes} % 'babel/polyglossia' detected but 'csquotes' missing.のWarningの回避
%%% Font shape `JY2/hmc/b/n' undefinedのWarningの回避
\DeclareFontShape{JY2}{hmc}{b}{n}{<->ssub*hmc/bx/n}{}
\DeclareFontShape{JT2}{hmc}{b}{n}{<->ssub*hmc/m/n}{}
\DeclareFontShape{JY2}{hmc}{m}{it}{<->ssub*hmc/m/n}{}
\DeclareFontShape{JT2}{hmc}{m}{it}{<->ssub*hmc/m/n}{}


% +++++++++++++++++++++++++++++++++++++++++++
% 本文
% +++++++++++++++++++++++++++++++++++++++++++

\begin{document}
\mojiparline{42}  % 一行あたり文字数
\linesparpage{38}  % 一ページあたり行数

\centerline{\textbf{
  修士論文のタイトル修士論文のタイトル修士論文のタイトル修士論文のタイトル
}}
\rightline{阪大~太郎}

% ===========================================
\section{緒言}
% ===========================================

\begin{wrapfigure}{r}{12zw}
  \vspace*{-\intextsep}
  \centering
  \includegraphics[width=10zw]{example-image-9x16}
  \caption{
    図のキャプション,図のキャプション.% 2022年度からは研究室内で日本語表記に統一する
  }
  \label{fig:introduction}
  \vspace*{-1.0zh} % 本文との間隔を調整するために\vspaceに適当な値(1zhは1行分,ptで指定もできる)を入れる
\end{wrapfigure}

\textcolor{LightGray}{
\jalipsum[1-2]{wagahai}
}


% ===========================================
\section{実験手法}  % 自身の研究に合わせて変更
% ===========================================

\begin{wrapfigure}{r}{26zw}
  \vspace*{-1.0zh} % 上に詰めて見出しの空白を有効活用する
  \centering
  \includegraphics[width=24zw]{example-image-16x9}
  \caption{
    図のキャプション,図のキャプション,図のキャプション,図のキャプション,図のキャプション.
  }
  \label{fig:experiment}
  \vspace*{-\intextsep}
\end{wrapfigure}

\textcolor{LightGray}{
\jalipsum[3]{wagahai}
\jalipsum[4]{wagahai}
}


% ===========================================
\section{実験結果}  % 自身の研究に合わせて変更
% ===========================================

\begin{figure}[htb]
  \centering
  \subfloat[\(f (\vb*{x},t)\)]{
    \includegraphics[width=0.24\textwidth]{example-image-1x1}
    \label{subfig:f}
  }
  \subfloat[\(g (\vb*{x},t)\)]{
    \includegraphics[width=0.24\textwidth]{example-image-1x1}
    \label{subfig:g}
  }
  \subfloat[\(h (\vb*{x},t)\)]{
    \includegraphics[width=0.24\textwidth]{example-image-1x1}
    \label{subfig:h}
  }
  \subfloat[\(i (\vb*{x},t)\)]{
    \includegraphics[width=0.24\textwidth]{example-image-1x1}
    \label{subfig:i}
  }
  \caption{
    図のキャプション,図のキャプション,図のキャプション,図のキャプション,図のキャプション,図のキャプション,図のキャプション.
  }
  \label{fig:fig_result}
  \vspace*{1.0zh}
\end{figure}

\textcolor{LightGray}{
\jalipsum[5]{wagahai}
}


% ===========================================
\section{結言}
% ===========================================

私は,〇〇〇という問題に注目し~\cite{Goto2017a,GotoJPS2018},図~\ref{fig:fig_result}のように〜〜〜を明らかにした~\cite{Araki_master_thesis}.% 2022年度からは研究室内でFig.ではなく図に統一する

% ===========================================
\printbibliography[title=参考文献]
% ===========================================

\end{document}
