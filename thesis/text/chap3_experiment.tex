\chapter{粒子画像流速測定による室内実験}
\label{chap:Experiment}

本章では,閉じた系の乱流を粒子画像流速測定(Particle Image Velocimetry,PIV)実験で測定した結果を報告する.
\ref{sec:ExperimentalSetup}節で実験装置の諸元についてまとめる.


\section{実験装置}
\label{sec:ExperimentalSetup}

実験装置の模式図を図~\ref{fig:experiment_schematic}に,実験に使用する機材の一覧を表~\ref{table:ListofEquipment}に示す.
\begin{figure}[!t]
  \centering
  \includegraphics[width=0.6\textwidth]{example-image-16x9}
  \caption{
    三次元CADによる実験装置の模式図.
  }
  \label{fig:experiment_schematic}
\end{figure}

\begin{table}[!t]
  \centering
  \caption{実験機材の一覧}
  \begin{tabular}{c|cc}
    機材名称 & メーカー & 型番 \\ \hline \hline
    円筒容器 & アクリル製 & 加工品 \\
    回転円盤 & 樹脂製 & 加工品 \\
    サーボモータ & オリエンタルモーター社製 & NX940AS--PS5 \\
    モータ制御用ソフトウェア & オリエンタルモーター社製 & MEXE02 \\
    ハイスピードカメラ & BASLER社製 & acA 1920--150 \si{\micro \meter} \\
    脱気装置 & ERC社製 & ERC--3302W \\
    ロッドレンズ & シグマ光機社製 & RODB--05L10 \\
    ミラー & シグマ光機社製 & TFA--10S05--10 \\
    プリズム & シグマ光機社製 & PRB2--10--550 \\
    平面球凸レンズ & シグマ光機社製 & SLB--30--800P
  \end{tabular}
  \label{table:ListofEquipment}
\end{table}


\subsection{系の無次元化}
\label{subsec:ES_UndimentionalizedSystem}

円盤の半径\(D/2\, \si{m}\)と回転円盤の外周速度\(U\, \si{m/s}\)よりReynolds数
\begin{equation}
  \Re \equiv \frac{UD/2}{\nu}
  \label{eq:ES_DefOfReynoldsNumber}
\end{equation}
を定義する.
ただし,外周速度\(U\)は円盤の回転周期\(T\, \si{s}\)と半径\(D/2\)をもちいて
\begin{equation}
  U \equiv \frac{2\pi D/2}{T}
  \label{eq:ES_DefOfVelocityAtRim}
\end{equation}
で表せる.
このとき,Reynolds数の定義は
\begin{empheq}{align}
  \Re &\equiv \frac{UD/2}{\nu} \nonumber \\
    &= \frac{\pi D^2}{2\nu T}
  \label{eq:ES_DefOfReynoldsNumber_T}
\end{empheq}
となる.
作動流体である水の動粘度は\(\nu = 1.0 \times 10^-6\, \si{m^2/s}\)とする.
実験で変化させるのは\(T\)のみであり,例えば\(T=2.0\, \si{s}\)のとき,Reynolds数は
\begin{empheq}{align}
  \Re &= \frac{\pi D^2}{2\nu T} \\
    &= \frac{\pi \times (0.200)^2 }{2 \times 1.0\times 10^{-6} \times 2.0} \fallingdotseq 3.1 \times 10^4
\end{empheq}
となる.
