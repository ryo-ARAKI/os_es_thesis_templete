\chapter*{付録}
\label{chap:Appendix}
\renewcommand{\thechapter}{A}  % 章番号をAに設定
\addcontentsline{toc}{chapter}{付録}  % 目次には表示

\begin{tcolorbox}
  本文に書くと冗長になる
  \begin{itemize}
    \item 実験の手順
    \item 数値計算や解析の(重要ではない)アルゴリズム
    \item 生データやより詳細な結果を示す図,表
  \end{itemize}
  などをここに執筆する.
  なお,本文とは独立した章番号としている.
\end{tcolorbox}


\section{PIVにおける誤ベクトル判定アルゴリズム}
\label{sec:PIV_ErrorVectorAlgorithm}

\subsection{中央値フィルタによる誤ベクトル除去}
\label{subsec:BOP_MedianFilter}

近傍流速ベクトルの中央値を用いたフィルタリングで誤ベクトルを除去する~\cite[p.161]{PIVhandbook}.
PIVによって格子状に得られた流速場のうち,\((i,j)\)座標における流速ベクトルを\(\vb*{u}_{i,j}\)と書く.
このベクトルの周囲に存在する最大8本の流速ベクトルの中央値ベクトル\(\vb*{u}_M\)を算出する.
このとき,閾値を\(\delta_\mathrm{th}=2\)として
\begin{equation}
  \abs{\vb*{u}_{i,j} - \vb*{u}_M} \ge \delta_\mathrm{th}
  \label{eq:BOP_MF_DefOfMedianFilter}
\end{equation}
となるものを誤ベクトルと判定する.
ただし,実装上はベクトル\(\vb*{u} = (u,v)\)の成分自乗和を用いて
\begin{empheq}{align}
  \abs{\vb*{u}_{i,j}}^2 - \abs{\vb*{u}_{i,j}}^2 &\equiv (u_{i,j}^2+v_{i,j}^2) - (u_{M:i,j}^2+v_{M:i,j}^2) \nonumber \\
    &\ge \delta_\mathrm{th}^2
  \label{eq:BOP_MF_ThresholdOfMedianFilter}
\end{empheq}
と計算している.
