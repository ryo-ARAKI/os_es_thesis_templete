\chapter*{記号表}

\begin{tcolorbox}
  この章では論文中で使用する記号を定義する.
  ここでの分類はテンプレート作成者が用いたものなので,自身で適切と思う分類に更新すること.
\end{tcolorbox}

\section*{パラメータ,無次元量}
\begin{tabular}{lll}
$\pi$  & : & 円周率 \\
$\nu$  & : & 動粘性係数 \\
$\rho$ & : & 密度 \\
$\mathrm{Re}$ & : & Reynolds数
\end{tabular}

\section*{物理量}
\begin{tabular}{lll}
$L$ & : & 乱流の特徴長さ \\
$U$ & : & 乱流の特徴速度 \\
$T$ & : & 乱流の特徴時間スケール \\ \hline
$\vb*{u}=(u,v,w)$  & : & 流速 \\
$p$ & : & 圧力 \\
$\epsilon$ & : & エネルギ散逸率 \\
$\varPi$ & : & エネルギ流束 \\ \hline
$\vb*{x}=(x,y,z)$ & : & 大域座標 \\
$\vb*{\xi}=(\xi_1,\xi_2)$ & : & 局所座標 \\
$t$ & : & 時刻
\end{tabular}

\begin{tcolorbox}
  ここで,物理量の中でも種類の違うものを \texttt{hline}で区切っている.
\end{tcolorbox}

\section*{関数,演算子}
\begin{tabular}{lll}
$Cor\qty[A,B]\qty(\alpha)$ & : & $A$と$B$の$\alpha$に関する相関関数 \\
$\delta_{ij}$ & : & Kroneckerのデルタ \\
$\expval{(\cdot)}$ & : & 空間平均 \\
$\overline{(\cdot)}$ & : & 時間平均 \\
$\mathcal{A}$ & : & 局所座標と大域座標を結ぶ写像
\end{tabular}

\section*{上付き・下付き文字}
\begin{tabular}{lll}
  $(\cdot)^e$ & : & 局所座標系の量 \\
  $(\cdot)_l \equiv \underline{(\cdot)^e}$ & : & 局所座標を並べた量 \\
  $(\cdot)_g \equiv \mathcal{A} (\cdot)_l$ & : & 大域座標の量 \\
\end{tabular}
